\documentclass[../main.tex]{subfiles}
\begin{document}
	The most important factor governing the spatial dynamics is peak power
	of pulse. For a Gaussian pulse, if peak power is more than the critical
	power then it will experience self focusing.
	\begin{equation} \label{eq:Pcr_gu}
		P_{cr} = \frac{1.22 \pi \lambda_0^2}{32 n_0 n_2}
	\end{equation}
	For BK7, $P_{cr}$ is $1.75 MW$.

	We have divided our study in different regimes.
	For input peak power from just above $P_{cr}$ to threshold
	power for the formation of plasma $P_{th}$, the dynamics are governed
	entirely by group velocity dispersion (GVD) and optical Kerr effect.
	Therefore we call this the \textbf{GVD regime}.

	If input peak power is more that $P_{th}$, the dominant mechanism to
	govern defocusing is not just GVD and SPM. To estimate self focusing
	distance correctly, we need to take effect of plasma creation into
	account. Self channelling with waveguids is a characteristic of this
	regime. We call this the \textbf{Ionization regime}.
\end{document}
