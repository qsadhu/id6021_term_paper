\documentclass[../main.tex]{subfiles}
\begin{document}
	If we talk about first principles, Maxwell's equations govern
	propagation of electromagnetic waves (EMWs) through a medium.
	\begin{equation} \label{eq:maxwell}
		\nabla \times \textbf{E} = - \frac{\partial \textbf{B}}{\partial t}
	\end{equation}
	\begin{equation}
		\nabla \times \textbf{B} = \mu_0 \left( \textbf{J} + \frac{\partial
		\textbf{D}}{\partial t} \right)
	\end{equation}
	where $\textbf{E}$ and $\textbf{B}$ denotes electric and magnetic
	fields. We can use well known method to come at wave equation from these
	Maxwell's equations [eq:~\ref{eq:maxwell}].
	\begin{equation} \label{eq:wave}
		- \nabla^2 \textbf{E}
		- \nabla \left( \nabla . \textbf{E} \right)
		+ \frac{1}{c^2} \frac{\partial^2 \textbf{D}}{\partial t^2}
		= \mu_0 \left( \frac{\partial \textbf{J}}{\partial t} +
		\frac{\partial^2 \textbf{P}}{\partial t^2} \right)
	\end{equation}
	It would be easy to work with [eq:~\ref{eq:wave}] if we write in in
	fourier space.
	\begin{equation} \label{eq:fourierwave}
		\nabla^2 \tilde{\textbf{E}} - \nabla(\nabla \dot
		\tilde{\textbf{E}}) +
		\frac{\omega^2 n^2(\omega)}{c^2} \tilde{\textbf{E}} = \mu_0
		(-i\omega \tilde{\textbf{J}} - \omega^2 \tilde{\textbf{P}})
	\end{equation}
	To make our work easier, we can assume fields to be linearly polarized
	along a direction transverse to propagation axis. We can also use
	relation between $\textbf{J}$ and
	$\textbf{P}$~\cite{couairon_practitioners_2011,
	noauthor_nonlinear_nodate} to come at
	\begin{equation} \label{eq:polar}
		(\partial_z^2 + \nabla_\perp^2)\tilde{E}(\textbf{r}, \omega, z)
		+ k^2(\omega) \tilde{E}(\textbf{r}, \omega, z) = -\mu_0 \omega^2
		\tilde{P}(\textbf{r}, \omega, z)
	\end{equation}
	It is convenient to shift wave equation in pulse frame for numerical
	simulations. If $v_g$ is group velocity of pulse, then
	eq:~\ref{eq:fourierwave} becomes
	\begin{equation} \label{eq:pufr}
		\frac{\partial^2 \tilde{E}}{\partial \zeta^2} + 2i
		\frac{\omega}{v_g} \frac{\partial \tilde{E}}{\partial \zeta} = -
		\Delta_\perp \tilde{E} - \left[k^2(\omega) -
		\frac{\omega^2}{v_g^2} \right] \tilde{E} - \frac{\omega^2}{c^2}
		\frac{\tilde{P}}{\epsilon_0}
	\end{equation}
	Where $\zeta = z$ and $\tau = t - z/v_g$.
	Taking only envelope and slowly varying envelope approximation
	(SVEA)~\cite{couairon_practitioners_2011},
	we could arrive at
	\begin{equation} \label{eq:canonical}
		\frac{\partial \mathcal{E}}{\partial \zeta} =
		\frac{i}{2n_0k_0}\Delta_\perp \mathcal{E} -
		i\frac{k_0^{(2)}}{2}
		\frac{\partial^2 \mathcal{E}}{\partial \tau^2} +
		\frac{i}{2n_0}\frac{\omega_0}{c}\frac{\mathcal{P}}{\epsilon_0}
	\end{equation}
	Eq:~\ref{eq:canonical} is called canonical form of Non-linear Schrödinger
	Equation (NLS). Generally $\mathcal{P}$ in [eq:~\ref{eq:canonical}] is
	taken for optical Kerr effect.
	\begin{equation} \label{eq:kerr}
		\frac{\mathcal{P}}{\epsilon_0} = 2n_0n_2\mathcal{I} \mathcal{E}
	\end{equation}
	where $\mathcal{I}$ is intensity and $n_2$ is nonlinear refractive
	index. We could also add higher order terms like $\partial_t^3 E$ to NLS
	equation. Subsituting $\mathcal{P}$ in [eq:~\ref{eq:canonical}],
	we get
	\begin{equation} \label{eq:nls}
		\frac{\partial \mathcal{E}}{\partial \zeta}
		+ i\frac{k_0^{(2)}}{2} \frac{\partial^2 \mathcal{E}}{\partial
		\tau^2} - \frac{i}{2n_0k_0}\Delta_\perp \mathcal{E}
		= i\frac{\omega_0}{c}n_2\mathcal{I} \mathcal{E}
	\end{equation}

	[Eq:~\ref{eq:nls}] models beam propagation under the effects of
	diffraction, dispersion and the optical Kerr effect, which leads to self
	focusing i.e cumulative lens effect due to higher refractive index in
	most intense part of the beam~\cite{couairon_practitioners_2011,
		couairon_femtosecond_2007}.
	The NLS equation is used to explain nonlinear effects in optical fiber
	in this form.
	In case of cylindrical symmetry beam
	collapsed under finite distance if power of beam is more than critical
	power.
	\begin{equation} \label{eq:Pcr}
		P_{cr} = \frac{3.77 \pi n_0}{2 k_0^2 n_2}
	\end{equation}
\end{document}
