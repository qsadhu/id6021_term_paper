\documentclass[../main.tex]{subfiles}
\begin{document}
The NLS equation [eq:~\ref{eq:nls}] leads to complete collapse in cylindrical
geometry, but in real experiment saturation mechanism prevent that to occur.
It shows us that [eq:~\ref{eq:nls}] could be improved for real experiments.
Beam intensity is high enough to cause nonlinear absorption when
it is close to collapse. Therefore we need to include some terms which capture
effect of local plasma created due to ionization of gases or transition of
electrons from valance band to conduction band~\cite{gopal_systematic_2007}.

In 1964, Keldysh~\cite{keldysh_ionization_1964} introduced a parameter
($\gamma$) to determine dominate mechanism of ionization.
\begin{equation} \label{eq:gamma}
	\gamma = \frac{\omega_0}{eE} (mU_i)^{1/2}
\end{equation}
Where $U_i$ is band gap and $E$ is electric field of laser.
If $\gamma > 1$ multi-photon ionization (MPI) is dominant and if $\gamma < 1$,
then tunnel ionization (TI) is dominant.
For intensity of laser of order of $10^{13}$, MPI is dominant.
Following Keldysh~\cite{keldysh_ionization_1964}, for MPI, intensity dependant
ionization rate $W(\mathcal{I})$ is directly perposnal to $\mathcal{I}^K$ where
$K$ is no of photon required for MPI.
\begin{equation} \label{eq:MPI_Rate}
	W(\mathcal{I}) = \sigma_K \mathcal{I}^K
\end{equation}
Nonlinear absorption could be described by an effective current $\textbf{J}$.
\begin{equation} \label{eq:eff_current}
	\frac{1}{2} \textbf{J}.\textbf{E}^* = W(\mathcal{I}) K \hbar \omega_0 \rho_{nt}
\end{equation}
\begin{equation}
	\frac{\textbf{J}}{\epsilon_0 c} = n_0 \frac{W(\mathcal{I}) K \hbar
	\omega_0 \rho_{nt}}{\mathcal{I}} \textbf{E}
\end{equation}
And in its envelope form, it can be written
\begin{equation} \label{eq:current}
	\frac{\mathcal{J}}{\epsilon_0 c} = n_0 \beta_K \mathcal{I}^{K-1}\mathcal{E}
\end{equation}
Where $\beta_K$ denote cross section of absorption for multiphoton absorption.
We could replace $\mathcal{P}$ to $\mathcal{P} +i{\mathcal{J}}/{\omega_0}$ and
introduce [eq:\ref{eq:current}] in NLS.
\begin{equation} \label{eq:nls_abs}
	\frac{\partial \mathcal{E}}{\partial \zeta}
		+ i\frac{k_0^{(2)}}{2} \frac{\partial^2 \mathcal{E}}{\partial
		\tau^2} - \frac{i}{2n_0k_0}\Delta_\perp \mathcal{E}
		= i\frac{\omega_0}{c}n_2\mathcal{I} \mathcal{E}
		- \frac{\beta_K}{2}\mathcal{I}^{K-1} \mathcal{E}
\end{equation}
\end{document}
